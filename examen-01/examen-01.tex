\documentclass[11pt]{beamer}
\usepackage{listings} % Include the listings-package
\usepackage[T1]{fontenc}
\usepackage[utf8]{inputenc}
\usepackage[english]{babel}
\usepackage{amsmath}
\usepackage{amssymb, amsfonts, latexsym, cancel}
\usepackage{float}
\usepackage{graphicx}
\usepackage{epstopdf}
\usepackage{subfigure}
\usepackage{hyperref}
%\usepackage{authblk}
\usepackage{blindtext}
\usepackage{booktabs} % Allows the use of \toprule, 
\usepackage{filecontents}
\usepackage{courier} %% Sets font for listing as Courier.
\usepackage{listings}
%\usepackage{listings, xcolor}
\lstset{
tabsize = 2, %% set tab space width
showstringspaces = false, %% prevent space marking in strings, string is defined as the text that is generally printed directly to the console
numbers = left, %% display line numbers on the left
commentstyle = \color{green}, %% set comment color
keywordstyle = \color{blue}, %% set keyword color
stringstyle = \color{red}, %% set string color
rulecolor = \color{black}, %% set frame color to avoid being affected by text color
basicstyle = \small \ttfamily , %% set listing font and size
breaklines = true, %% enable line breaking
numberstyle = \tiny,
}
\usepackage{caption}
\DeclareCaptionFont{white}{\color{white}}
\DeclareCaptionFormat{listing}{\colorbox{gray}{\parbox{\textwidth}{#1#2#3}}}
\captionsetup[lstlisting]{format=listing,labelfont=white,textfont=white}
\definecolor{urlColor}{rgb}{0.06, 0.3, 0.57}
\definecolor{linkColor}{rgb}{0.57, 0.0, 0.04}
\definecolor{fileColor}{rgb}{0.0, 0.26, 0.26}
\hypersetup{
    colorlinks=true,
    linkcolor=linkColor,
    filecolor=fileColor,      
    urlcolor=urlColor,
}
\urlstyle{same}
\setbeamercovered{transparent}
%\usetheme{Boadilla}
\usetheme{CambridgeUS}
%\usetheme{Berkeley}
%\usetheme{Warsaw}
%\usetheme{Madrid}

\title[Introducción]{\bf\Huge Pautas de diseño de la interfaz de usuario}
\subtitle{Interacción Humano Computador}

\author[rescobedoq]
{
	Jeampier Anderson Moran Fuño \inst{1}\\
	Marcelo Andre Guevara Guitierrez\inst{1}\\
	Rony Tito Ventura Ramos\inst{1}\\
	Rudy Roberto Tito Durand\inst{1}
}
\institute[UNSA]
{
\inst{1}% 
System Engineering School\\
System Engineering and Informatic Department\\
Production and Services Faculty\\
San Agustin National University of Arequipa
}

\date[2020-15-09]{\scriptsize{2020-15-09}}
%\logo{\includegraphics[width=3.0cm]{img/logo_unsa.jpg}}
\titlegraphic{\includegraphics[width=1.0cm]{img/logo_unsa.jpg}}

\begin{document}

\begin{frame}
\titlepage
\end{frame}

\begin{frame}
\frametitle{Content}
\tableofcontents
\end{frame}

\section{Aqui Pondremos la primera Diapo}
\begin{frame}
\frametitle{Introducción}
Los videojuegos son una de las principales formas de entretenimiento entre niños y jóvenes y su industria se ha desarrollado a la par del desarrollo de las tecnologías de la información. El crecimiento de la industria de los videojuegos en las últimas décadas ha sido tal que ya ha superado la facturación del cine y la música juntos. Además de su importancia económica, algunos videojuegos han sido denominados como obras de arte y muchos consideran que el desarrollo y diseño de videojuegos debería ser considerado el octavo arte, ya que se necesita mucha imaginación e inspiración para crear un universo que se puede usar para desarrollar un videojuego .

La innovación y el desarrollo tecnológico están estrechamente relacionados con el desarrollo de los videojuegos y cada vez nos sorprende más la estética y la jugabilidad de los últimos juegos lanzados.

\end{frame}

\end{document}
