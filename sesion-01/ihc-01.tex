\documentclass[11pt]{beamer}
\usepackage{listings} % Include the listings-package
\usepackage[T1]{fontenc}
\usepackage[utf8]{inputenc}
\usepackage[english]{babel}
\usepackage{amsmath}
\usepackage{amssymb, amsfonts, latexsym, cancel}
\usepackage{float}
\usepackage{graphicx}
\usepackage{epstopdf}
\usepackage{subfigure}
\usepackage{hyperref}
%\usepackage{authblk}
\usepackage{blindtext}
\usepackage{booktabs} % Allows the use of \toprule, 
\usepackage{filecontents}
\usepackage{courier} %% Sets font for listing as Courier.
\usepackage{listings}
%\usepackage{listings, xcolor}
\lstset{
tabsize = 2, %% set tab space width
showstringspaces = false, %% prevent space marking in strings, string is defined as the text that is generally printed directly to the console
numbers = left, %% display line numbers on the left
commentstyle = \color{green}, %% set comment color
keywordstyle = \color{blue}, %% set keyword color
stringstyle = \color{red}, %% set string color
rulecolor = \color{black}, %% set frame color to avoid being affected by text color
basicstyle = \small \ttfamily , %% set listing font and size
breaklines = true, %% enable line breaking
numberstyle = \tiny,
}
\usepackage{caption}
\DeclareCaptionFont{white}{\color{white}}
\DeclareCaptionFormat{listing}{\colorbox{gray}{\parbox{\textwidth}{#1#2#3}}}
\captionsetup[lstlisting]{format=listing,labelfont=white,textfont=white}
\definecolor{urlColor}{rgb}{0.06, 0.3, 0.57}
\definecolor{linkColor}{rgb}{0.57, 0.0, 0.04}
\definecolor{fileColor}{rgb}{0.0, 0.26, 0.26}
\hypersetup{
    colorlinks=true,
    linkcolor=linkColor,
    filecolor=fileColor,      
    urlcolor=urlColor,
}
\urlstyle{same}
\setbeamercovered{transparent}
%\usetheme{Boadilla}
\usetheme{CambridgeUS}
%\usetheme{Berkeley}
%\usetheme{Warsaw}
%\usetheme{Madrid}

\title[Introducción]{\bf\Huge Pautas de diseño de la interfaz de usuario}
\subtitle{Interacción Humano Computador}

\author[rescobedoq]
{
	Jeampier Anderson Moran Fuño \inst{1}\\
	Marcelo Andre Guevara Guitierrez\inst{1}\\
	Rony Tito Ventura Ramos\inst{1}\\
	Rudy Roberto Tito Durand\inst{1}
}
\institute[UNSA]
{
\inst{1}% 
System Engineering School\\
System Engineering and Informatic Department\\
Production and Services Faculty\\
San Agustin National University of Arequipa
}

\date[2020-15-09]{\scriptsize{2020-15-09}}
%\logo{\includegraphics[width=3.0cm]{img/logo_unsa.jpg}}
\titlegraphic{\includegraphics[width=1.0cm]{img/logo_unsa.jpg}}

\begin{document}

\begin{frame}
\titlepage
\end{frame}

\begin{frame}
\frametitle{Content}

\tableofcontents
\end{frame}

\section{Koyani ,breve biografia}
\begin{frame}
\frametitle{Koyani ,breve biografia}
El Sr. Koyani tiene 20 años de experiencia en comunicaciones estratégicas, operaciones de programas, metodologías de innovación, datos y soluciones de TI y planificación estratégica. Actualmente, el Sr. Koyani se desempeña como Director Ejecutivo de Innovación en la Oficina Inmediata del Secretario del HHS y proporciona liderazgo en los programas de innovación en todo el HHS que permiten al HHS mejorar los resultados de la atención médica y modernizar las operaciones gubernamentales. Además, el Sr. Koyani lidera la iniciativa de optimización del desempeño regional del HHS. Esta prioridad de la Secretaría del HHS se centra en la modernización de las 10 oficinas regionales del HHS ubicadas en los EE. UU. En función de las prioridades de la agencia, los datos y las necesidades de las partes interesadas.
\end{frame}

\section{Koyani ,breve biografia 2}
\begin{frame}
\frametitle{Koyani ,breve biografia}
Antes de esto, el Sr. Koyani dirigió programas importantes en comunicación innovadora y estrategia web, supervisión presupuestaria y del Congreso, y estableció operaciones y sistemas innovadores para nuevas Oficinas como la primera División de Comunicaciones Web del HHS y el Centro de Productos de Tabaco de la FDA.
\end{frame}

\section{Pautas de diseño de la interfaz de usuario}
\begin{frame}
\frametitle{Pautas de diseño de la interfaz de usuario}
\begin{itemize}
\item Sistemas informáticos interactivos
\item Pautas de diseño de la interfaz de usuario. (Reglas de diseño):
\item Cheriton (1976) - Tiempo compartido.
\item Norman (1983) - Cognición humana.
\item Smith y Mosier (1986).
\item Shneiderman (1987) - Ocho reglas de oro.
\item Brown (1988).
\item Nielsen y Molich (1990) - Evaluación heurística.
\item Nielsen y Mack (1994).
\item Stone et al. (2005).
\item {\bf Koyani y col. (2006).}
\item Johnson (2007).
\item Shneiderman y Plaisant (2009).
\item Microsoft, Apple Computer y Oracle - 2009, 2009, 2001
\end{itemize}
\end{frame}

\section{Koyani and col}
\begin{frame}

\frametitle{Guia para el diseño web, Koyani}
\begin{itemize}
\item Mostrar todas las opciones en la pagina principal.
\item Informar los tiempo de sesion.
\item Utilizar terminologia simple en los documentos de ayuda.
\item Proveer asistencia a los usuarios.
\item Comunicar el proposito y valor del sitio web.
\item Anunciar los futuros cambios del sitio web.
\item Añadir paneles de ayuda en la pagina principal.
\end{itemize}

\end{frame}

\section{References}
%References frame
\begin{frame}
\frametitle{References}
\begin{itemize}
\item Introduction - xiii, Jhonson J. (2014). Designing with the Mind in mind. 2nd. edition.
\item Koyani,  S.  J.,  Bailey,  R.  W.,  and  Nall,  J.  R.  (2006). Research-based  web  design  and  usability  guide-lines. US Department of Health and Human Services. Website: usability.gov/pdfs/guidelines.html. 
\end{itemize}
\end{frame}

\end{document}
